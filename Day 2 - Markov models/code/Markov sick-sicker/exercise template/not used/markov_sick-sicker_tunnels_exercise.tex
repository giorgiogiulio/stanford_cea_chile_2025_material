% Options for packages loaded elsewhere
\PassOptionsToPackage{unicode}{hyperref}
\PassOptionsToPackage{hyphens}{url}
%
\documentclass[
]{article}
\usepackage{lmodern}
\usepackage{amssymb,amsmath}
\usepackage{ifxetex,ifluatex}
\ifnum 0\ifxetex 1\fi\ifluatex 1\fi=0 % if pdftex
  \usepackage[T1]{fontenc}
  \usepackage[utf8]{inputenc}
  \usepackage{textcomp} % provide euro and other symbols
\else % if luatex or xetex
  \usepackage{unicode-math}
  \defaultfontfeatures{Scale=MatchLowercase}
  \defaultfontfeatures[\rmfamily]{Ligatures=TeX,Scale=1}
\fi
% Use upquote if available, for straight quotes in verbatim environments
\IfFileExists{upquote.sty}{\usepackage{upquote}}{}
\IfFileExists{microtype.sty}{% use microtype if available
  \usepackage[]{microtype}
  \UseMicrotypeSet[protrusion]{basicmath} % disable protrusion for tt fonts
}{}
\makeatletter
\@ifundefined{KOMAClassName}{% if non-KOMA class
  \IfFileExists{parskip.sty}{%
    \usepackage{parskip}
  }{% else
    \setlength{\parindent}{0pt}
    \setlength{\parskip}{6pt plus 2pt minus 1pt}}
}{% if KOMA class
  \KOMAoptions{parskip=half}}
\makeatother
\usepackage{xcolor}
\IfFileExists{xurl.sty}{\usepackage{xurl}}{} % add URL line breaks if available
\IfFileExists{bookmark.sty}{\usepackage{bookmark}}{\usepackage{hyperref}}
\hypersetup{
  pdftitle={Decision Modeling for Public Health},
  pdfauthor={The DARTH workgroup},
  hidelinks,
  pdfcreator={LaTeX via pandoc}}
\urlstyle{same} % disable monospaced font for URLs
\usepackage[margin=1in]{geometry}
\usepackage{longtable,booktabs}
% Correct order of tables after \paragraph or \subparagraph
\usepackage{etoolbox}
\makeatletter
\patchcmd\longtable{\par}{\if@noskipsec\mbox{}\fi\par}{}{}
\makeatother
% Allow footnotes in longtable head/foot
\IfFileExists{footnotehyper.sty}{\usepackage{footnotehyper}}{\usepackage{footnote}}
\makesavenoteenv{longtable}
\usepackage{graphicx,grffile}
\makeatletter
\def\maxwidth{\ifdim\Gin@nat@width>\linewidth\linewidth\else\Gin@nat@width\fi}
\def\maxheight{\ifdim\Gin@nat@height>\textheight\textheight\else\Gin@nat@height\fi}
\makeatother
% Scale images if necessary, so that they will not overflow the page
% margins by default, and it is still possible to overwrite the defaults
% using explicit options in \includegraphics[width, height, ...]{}
\setkeys{Gin}{width=\maxwidth,height=\maxheight,keepaspectratio}
% Set default figure placement to htbp
\makeatletter
\def\fps@figure{htbp}
\makeatother
\setlength{\emergencystretch}{3em} % prevent overfull lines
\providecommand{\tightlist}{%
  \setlength{\itemsep}{0pt}\setlength{\parskip}{0pt}}
\setcounter{secnumdepth}{-\maxdimen} % remove section numbering

\title{Decision Modeling for Public Health}
\usepackage{etoolbox}
\makeatletter
\providecommand{\subtitle}[1]{% add subtitle to \maketitle
  \apptocmd{\@title}{\par {\large #1 \par}}{}{}
}
\makeatother
\subtitle{Markov Model Variants Exercise}
\author{The DARTH workgroup}
\date{}

\begin{document}
\maketitle

Developed by the Decision Analysis in R for Technologies in Health
(DARTH) workgroup:

Fernando Alarid-Escudero, PhD (1)

Eva A. Enns, MS, PhD (2)

M.G. Myriam Hunink, MD, PhD (3,4)

Hawre J. Jalal, MD, PhD (5)

Eline M. Krijkamp, MSc (3)

Petros Pechlivanoglou, PhD (6,7)

Alan Yang, MSc (7)

In collaboration of:

\begin{enumerate}
\def\labelenumi{\arabic{enumi}.}
\tightlist
\item
  Division of Public Administration, Center for Research and Teaching in
  Economics (CIDE), Aguascalientes, Mexico
\item
  University of Minnesota School of Public Health, Minneapolis, MN, USA
\item
  Erasmus MC, Rotterdam, The Netherlands
\item
  Harvard T.H. Chan School of Public Health, Boston, USA
\item
  University of Pittsburgh Graduate School of Public Health, Pittsburgh,
  PA, USA
\item
  University of Toronto, Toronto ON, Canada
\item
  The Hospital for Sick Children, Toronto ON, Canada
\end{enumerate}

Please cite our publications when using this code:

\begin{itemize}
\item
  Jalal H, Pechlivanoglou P, Krijkamp E, Alarid-Escudero F, Enns E,
  Hunink MG. An Overview of R in Health Decision Sciences. Med Decis
  Making. 2017; 37(3): 735-746.
  \url{https://journals.sagepub.com/doi/abs/10.1177/0272989X16686559}
\item
  Alarid-Escudero F, Krijkamp EM, Enns EA, Yang A, Hunink MGM
  Pechlivanoglou P, Jalal H. Cohort State-Transition Models in R: A
  Tutorial. arXiv:200107824v2. 2020:1-48.
  \url{http://arxiv.org/abs/2001.07824}
\item
  Krijkamp EM, Alarid-Escudero F, Enns EA, Jalal HJ, Hunink MGM,
  Pechlivanoglou P. Microsimulation modeling for health decision
  sciences using R: A tutorial. Med Decis Making. 2018;38(3):400--22.
  \url{https://journals.sagepub.com/doi/abs/10.1177/0272989X18754513}
\item
  Krijkamp EM, Alarid-Escudero F, Enns E, Pechlivanoglou P, Hunink MM,
  Jalal H. A Multidimensional Array Representation of State-Transition
  Model Dynamics. Med Decis Making. 2020 Online first.
  \url{https://doi.org/10.1177/0272989X19893973}
\end{itemize}

Copyright 2017, THE HOSPITAL FOR SICK CHILDREN AND THE COLLABORATING
INSTITUTIONS. All rights reserved in Canada, the United States and
worldwide. Copyright, trademarks, trade names and any and all associated
intellectual property are exclusively owned by THE HOSPITAL FOR Sick
CHILDREN and the collaborating institutions. These materials may be
used, reproduced, modified, distributed and adapted with proper
attribution.

\hypertarget{exercise-i-variations-on-the-sick-sicker-markov-model}{%
\section{Exercise I: Variations on the Sick-Sicker Markov
Model}\label{exercise-i-variations-on-the-sick-sicker-markov-model}}

Previously, you built a Markov of the Sick-Sicker model where transition
probabilities were assumed to be constant over time. In this exercise,
you will expand on that model to incorporate dependence on time since
model start and on state residence.

\textbf{Time since model start}

Healthy individuals are no longer assumed to have a fixed mortality
rate. Their mortality rate depends on their age. The Human Mortality
Database (HMD) (www.mortality.org) provides these age specific mortality
rates (Mx1x1). You can load this data in R using different packages, for
example the HMDHFplus, demography or data.table package. Registration
(free) on the HMD website is required in order to use the data. For this
exercise, we provide you with this mortality rate for the US in the file
``HMD\_USA\_Mx\_2015.csv''.

Remember: individuals in S1 and S2 still have an increased mortality
relative to healthy individuals, as described in the original exercise.
The same hazard ratios are used to calculate the probabilities of dying
from S1 and S2 in this exercise.

\textbf{State residence}

It has been recently discovered that the risk of progression from Sick
to Sicker increases the longer a person has been sick. This increase
follows a Weibull growth curve, calculated as

\(p_{S1S2(t)} = \lambda_\gamma t^{(\gamma-1)}\)

where \(t\) is the \(t\)-th cycle (year) that a person has been in the
Sick state. \(\lambda = 0.08\) and \(\gamma = 1.1\) are the scale and
shape parameters of the Weibull function, respectively.

We will now expand the model to include age and state residence
dependency by adding age varying probabilies for death and tunnel states
for S1, as shown in Figure 2.

\begin{figure}

{\centering \includegraphics[width=1\linewidth]{C:/Users/Alan Yang/Desktop/GitHub local/Course-Modularization/static/Course_Modularization/Markov models/Markov Sick-Sicker/figures/sick_sicker_diagram} 

}

\caption{Schematic representation of the Sick-Sicker model}\label{fig:unnamed-chunk-1}
\end{figure}

\begin{figure}

{\centering \includegraphics[width=1\linewidth]{C:/Users/Alan Yang/Desktop/GitHub local/Course-Modularization/static/Course_Modularization/Markov models/Markov Sick-Sicker/figures/sick_sicker_tunnels_diagram} 

}

\caption{Schematic representation of the Sick-Sicker model with tunnels states for S1}\label{fig:unnamed-chunk-2}
\end{figure}

\hypertarget{tasks}{%
\subsection{Tasks}\label{tasks}}

Using the template \texttt{markov\_sick-sicker\_tunnels\_template.Rmd},
please do the following.

\begin{enumerate}
\def\labelenumi{\arabic{enumi}.}
\item
  Incorporate the tunnel states in the Markov trace and initialize it
  with everyone being healthy at model start.
\item
  Create a 3D transition probability array to account for tunnels and
  age dependency.
\item
  Fill in the 3D transition probability array accounting for the tunnel
  states for S1 and the age dependence of transitioning to death
\item
  Costs and utilities for all tunnel states are the same. Therefore,
  aggregate the Markov trace back to a healthy sick-sicker-dead trace
  and estimate total costs and QALYs for both strategies.
\item
  Estimate incremental costs and QALYs and the ICER.
\item
  Plot the survival curve for the cohort under no treatment.
\end{enumerate}

\begin{longtable}[]{@{}llc@{}}
\toprule
\begin{minipage}[b]{0.51\columnwidth}\raggedright
\textbf{Parameter}\strut
\end{minipage} & \begin{minipage}[b]{0.19\columnwidth}\raggedright
\textbf{R name}\strut
\end{minipage} & \begin{minipage}[b]{0.21\columnwidth}\centering
\textbf{Value}\strut
\end{minipage}\tabularnewline
\midrule
\endhead
\begin{minipage}[t]{0.51\columnwidth}\raggedright
Time horizon\strut
\end{minipage} & \begin{minipage}[t]{0.19\columnwidth}\raggedright
\texttt{n\_t}\strut
\end{minipage} & \begin{minipage}[t]{0.21\columnwidth}\centering
30 years\strut
\end{minipage}\tabularnewline
\begin{minipage}[t]{0.51\columnwidth}\raggedright
Cycle length\strut
\end{minipage} & \begin{minipage}[t]{0.19\columnwidth}\raggedright
\strut
\end{minipage} & \begin{minipage}[t]{0.21\columnwidth}\centering
1 year\strut
\end{minipage}\tabularnewline
\begin{minipage}[t]{0.51\columnwidth}\raggedright
Names of health states\strut
\end{minipage} & \begin{minipage}[t]{0.19\columnwidth}\raggedright
\texttt{v\_n}\strut
\end{minipage} & \begin{minipage}[t]{0.21\columnwidth}\centering
H, S1, S2, D\strut
\end{minipage}\tabularnewline
\begin{minipage}[t]{0.51\columnwidth}\raggedright
Annual discount rate (costs/QALYs)\strut
\end{minipage} & \begin{minipage}[t]{0.19\columnwidth}\raggedright
\texttt{d\_c} \texttt{d\_e}\strut
\end{minipage} & \begin{minipage}[t]{0.21\columnwidth}\centering
3\%\strut
\end{minipage}\tabularnewline
\begin{minipage}[t]{0.51\columnwidth}\raggedright
Annual transition probabilities\strut
\end{minipage} & \begin{minipage}[t]{0.19\columnwidth}\raggedright
\strut
\end{minipage} & \begin{minipage}[t]{0.21\columnwidth}\centering
\strut
\end{minipage}\tabularnewline
\begin{minipage}[t]{0.51\columnwidth}\raggedright
- Disease onset (H to S1), conditional on surviving\strut
\end{minipage} & \begin{minipage}[t]{0.19\columnwidth}\raggedright
\texttt{p\_HS1}\strut
\end{minipage} & \begin{minipage}[t]{0.21\columnwidth}\centering
0.15\strut
\end{minipage}\tabularnewline
\begin{minipage}[t]{0.51\columnwidth}\raggedright
- Recovery (S1 to H), conditional on surviving\strut
\end{minipage} & \begin{minipage}[t]{0.19\columnwidth}\raggedright
\texttt{p\_S1H}\strut
\end{minipage} & \begin{minipage}[t]{0.21\columnwidth}\centering
0.5\strut
\end{minipage}\tabularnewline
\begin{minipage}[t]{0.51\columnwidth}\raggedright
- Disease progression (S1 to S2), conditional on surviving\strut
\end{minipage} & \begin{minipage}[t]{0.19\columnwidth}\raggedright
\texttt{p\_S1S2}\strut
\end{minipage} & \begin{minipage}[t]{0.21\columnwidth}\centering
Weibull function\strut
\end{minipage}\tabularnewline
\begin{minipage}[t]{0.51\columnwidth}\raggedright
Annual mortality\strut
\end{minipage} & \begin{minipage}[t]{0.19\columnwidth}\raggedright
\strut
\end{minipage} & \begin{minipage}[t]{0.21\columnwidth}\centering
\strut
\end{minipage}\tabularnewline
\begin{minipage}[t]{0.51\columnwidth}\raggedright
- All-cause mortality (H to D)\strut
\end{minipage} & \begin{minipage}[t]{0.19\columnwidth}\raggedright
\texttt{p\_HD}\strut
\end{minipage} & \begin{minipage}[t]{0.21\columnwidth}\centering
Age-dependent\strut
\end{minipage}\tabularnewline
\begin{minipage}[t]{0.51\columnwidth}\raggedright
- Hazard ratio of death in S1 vs H\strut
\end{minipage} & \begin{minipage}[t]{0.19\columnwidth}\raggedright
\texttt{hr\_S1}\strut
\end{minipage} & \begin{minipage}[t]{0.21\columnwidth}\centering
3\strut
\end{minipage}\tabularnewline
\begin{minipage}[t]{0.51\columnwidth}\raggedright
- Hazard ratio of death in S2 vs H\strut
\end{minipage} & \begin{minipage}[t]{0.19\columnwidth}\raggedright
\texttt{hr\_S2}\strut
\end{minipage} & \begin{minipage}[t]{0.21\columnwidth}\centering
10\strut
\end{minipage}\tabularnewline
\begin{minipage}[t]{0.51\columnwidth}\raggedright
Annual costs\strut
\end{minipage} & \begin{minipage}[t]{0.19\columnwidth}\raggedright
\strut
\end{minipage} & \begin{minipage}[t]{0.21\columnwidth}\centering
\strut
\end{minipage}\tabularnewline
\begin{minipage}[t]{0.51\columnwidth}\raggedright
- Healthy individuals\strut
\end{minipage} & \begin{minipage}[t]{0.19\columnwidth}\raggedright
\texttt{c\_H}\strut
\end{minipage} & \begin{minipage}[t]{0.21\columnwidth}\centering
\$2,000\strut
\end{minipage}\tabularnewline
\begin{minipage}[t]{0.51\columnwidth}\raggedright
- Sick individuals in S1\strut
\end{minipage} & \begin{minipage}[t]{0.19\columnwidth}\raggedright
\texttt{c\_S1}\strut
\end{minipage} & \begin{minipage}[t]{0.21\columnwidth}\centering
\$4,000\strut
\end{minipage}\tabularnewline
\begin{minipage}[t]{0.51\columnwidth}\raggedright
- Sick individuals in S2\strut
\end{minipage} & \begin{minipage}[t]{0.19\columnwidth}\raggedright
\texttt{c\_S2}\strut
\end{minipage} & \begin{minipage}[t]{0.21\columnwidth}\centering
\$15,000\strut
\end{minipage}\tabularnewline
\begin{minipage}[t]{0.51\columnwidth}\raggedright
- Dead individuals\strut
\end{minipage} & \begin{minipage}[t]{0.19\columnwidth}\raggedright
\texttt{c\_D}\strut
\end{minipage} & \begin{minipage}[t]{0.21\columnwidth}\centering
\$0\strut
\end{minipage}\tabularnewline
\begin{minipage}[t]{0.51\columnwidth}\raggedright
- Additional costs of sick individuals treated in S1 or S2\strut
\end{minipage} & \begin{minipage}[t]{0.19\columnwidth}\raggedright
\texttt{c\_trt}\strut
\end{minipage} & \begin{minipage}[t]{0.21\columnwidth}\centering
\$12,000\strut
\end{minipage}\tabularnewline
\begin{minipage}[t]{0.51\columnwidth}\raggedright
Utility weights\strut
\end{minipage} & \begin{minipage}[t]{0.19\columnwidth}\raggedright
\strut
\end{minipage} & \begin{minipage}[t]{0.21\columnwidth}\centering
\strut
\end{minipage}\tabularnewline
\begin{minipage}[t]{0.51\columnwidth}\raggedright
- Healthy individuals\strut
\end{minipage} & \begin{minipage}[t]{0.19\columnwidth}\raggedright
\texttt{u\_H}\strut
\end{minipage} & \begin{minipage}[t]{0.21\columnwidth}\centering
1.00\strut
\end{minipage}\tabularnewline
\begin{minipage}[t]{0.51\columnwidth}\raggedright
- Sick individuals in S1\strut
\end{minipage} & \begin{minipage}[t]{0.19\columnwidth}\raggedright
\texttt{u\_S1}\strut
\end{minipage} & \begin{minipage}[t]{0.21\columnwidth}\centering
0.75\strut
\end{minipage}\tabularnewline
\begin{minipage}[t]{0.51\columnwidth}\raggedright
- Sick individuals in S2\strut
\end{minipage} & \begin{minipage}[t]{0.19\columnwidth}\raggedright
\texttt{u\_S2}\strut
\end{minipage} & \begin{minipage}[t]{0.21\columnwidth}\centering
0.50\strut
\end{minipage}\tabularnewline
\begin{minipage}[t]{0.51\columnwidth}\raggedright
- Dead individuals\strut
\end{minipage} & \begin{minipage}[t]{0.19\columnwidth}\raggedright
\texttt{u\_D}\strut
\end{minipage} & \begin{minipage}[t]{0.21\columnwidth}\centering
0.00\strut
\end{minipage}\tabularnewline
\begin{minipage}[t]{0.51\columnwidth}\raggedright
Intervention effect\strut
\end{minipage} & \begin{minipage}[t]{0.19\columnwidth}\raggedright
\strut
\end{minipage} & \begin{minipage}[t]{0.21\columnwidth}\centering
\strut
\end{minipage}\tabularnewline
\begin{minipage}[t]{0.51\columnwidth}\raggedright
- Utility for treated individuals in S1\strut
\end{minipage} & \begin{minipage}[t]{0.19\columnwidth}\raggedright
\texttt{u\_trt}\strut
\end{minipage} & \begin{minipage}[t]{0.21\columnwidth}\centering
0.95\strut
\end{minipage}\tabularnewline
\bottomrule
\end{longtable}

*Note: To calculate the probability of dying from S1 and S2, use the
hazard ratios provided. To do so, first convert the probability of dying
from healthy, \texttt{p\_HD}, to a rate; then multiply this rate by the
appropriate hazard ratio; finally, convert this rate back to a
probability. Recall that you can convert between rates and probabilities
using the following formulas: \(r = -log(1-p)\) and \(p = 1-e^{(-rt)}\)

\end{document}
